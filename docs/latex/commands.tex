%
% Diese Datei enthält einige Befehlsdefinitionen
%
%%%%%%%%%%%%%%%%%%%%
% Wörtliche Zitate %
%%%%%%%%%%%%%%%%%%%%
%
% Zitat ohne Referenz auf die Seitenangabe
\newcommand{\zitat}[2]{\glqq{}\textsl{#1}\grqq{} \refbib{#2}}
\newcommand{\zitatGross}[2]{\begin{quote}\glqq{}\textsl{#1}\grqq{} \refbib{#2}\end{quote}}
\newcommand{\zitatRiesig}[2]{\begin{quotation}\glqq{}\textsl{#1}\grqq{} \refbib{#2}\end{quotation}}
%
% Zitat mit Referenz auf die Seitenangabe
\newcommand{\zitatSeite}[3]{\glqq{}\textsl{#1}\grqq{} \refbibpage{#3}{#2}}
\newcommand{\zitatGrossSeite}[3]{\begin{quote}\glqq{}\textsl{#1}\grqq{} \refbibpage{#3}{#2}\end{quote}}
\newcommand{\zitatRiesigSeite}[3]{\begin{quotation}\glqq{}\textsl{#1}\grqq{} \refbibpage{#3}{#2}\end{quotation}}
%
% Einschübe, Auslassungen und erkannte Fehler in Zitaten
\newcommand{\auslassung}{[\ldots] }
\newcommand{\einschub}[1]{[#1] }
\newcommand{\erkannterFehler}{\einschub{sic!}}
%
%%%%%%%%%%%%%%%%%%
% Formatierungen %
%%%%%%%%%%%%%%%%%%
%
% Formatierungen für Acronyme
\newcommand{\acronym}[1]{{{#1}}}
%
%%%%%%%%%%%%%%%%%%%%%%%%%%%%%%%%%%
% Zu ändernd Stellen im Dokument %
% während der Bearbeitung        %
%%%%%%%%%%%%%%%%%%%%%%%%%%%%%%%%%%
%
\newcommand{\fixme}[1]{{\sffamily\bfseries{\Huge\uppercase{FIXME}}}{\large - #1}\\}
\newcommand{\frage}[1]{{\sffamily\bfseries{\Huge\scshape\uppercase{FRAGEN}}}{\large - #1}\\}
%
%%%%%%%%%%%%%%%%%%%%%%%%%%%%%%%%%%%%%%%%%%%%%%%%%%%%
% Befehle für ein Literaturverzeichnis ohne BibTEX %
%%%%%%%%%%%%%%%%%%%%%%%%%%%%%%%%%%%%%%%%%%%%%%%%%%%%
%
% Quellen aus dem Internet
\newcommand{\BIBdeWWWorg}[6]{\bibitem[#1]{#2}#3.\ \glqq{}#4\grqq{}.\\\href{#5}{#5}.\\#6.}
\newcommand{\BIBenWWWorg}[6]{\BIBdeWWWorg{#1}{#2}{#3}{\selectlanguage{english}#4\selectlanguage{ngerman}}{#5}{#6}}
\newcommand{\BIBenWWWauthor}[7]{\BIBenWWWorg{#1}{#2}{#3,\ #4}{#5}{#6}{#7}}
\newcommand{\BIBdeWWWauthor}[7]{\BIBdeWWWorg{#1}{#2}{#3,\ #4}{#5}{#6}{#7}}
%
% Use with long URLs, which have to be layouted manually.
% These commands use an additional last argument for a formatted URL
\newcommand{\BIBdeWWWorgLONG}[7]{\bibitem[#1]{#2}#3.\ \glqq{}#4\grqq{}.\\\href{#5}{#7}.\\#6.}
\newcommand{\BIBenWWWorgLONG}[7]{\BIBdeWWWorgLONG{#1}{#2}{#3}{\selectlanguage{english}#4\selectlanguage{ngerman}}{#5}{#6}{#7}}
\newcommand{\BIBenWWWauthorLONG}[8]{\BIBenWWWorgLONG{#1}{#2}{#3,\ #4}{#5}{#6}{#7}{#8}}
\newcommand{\BIBdeWWWauthorLONG}[8]{\BIBdeWWWorgLONG{#1}{#2}{#3,\ #4}{#5}{#6}{#7}{#8}}
%
%
% Für Freunde des ungenauen Zitierstils: Makro zum Ein- oder Ausblenden von Seitenzahlen
% Je nach Wunsch auskommentieren
% 
\newcommand{\refbib}[1]{\cite{#1}}
% Mit Seitenzahlen 
%\newcommand{\refbibpage}[2]{\cite[#1]{#2}}
% Ohne Seitenzahlen
\newcommand{\refbibpage}[2]{\cite{#2}}
%
%%%%%%%%%%%%%%%%%%%%%%%%
% Lokale Erweiterungen %
% für dieses Dokument  %
%%%%%%%%%%%%%%%%%%%%%%%%
%
% Befehle für die Verlinkung ummerhalb des Dokuments
% \link{text}{link-label}{Kapitel-Name}
\newcommand{\link}[3]{\emph{#1}\footnote{Siehe: \ref{#2} {#3}}}
\newcommand{\citeLink}[3]{\emph{#1}$^($\footnote{Siehe: \ref{#2} {#3}}$^)$}
%
% Häufig verwendete Akronyme
\newcommand{\UML}{\acronym{UML}}
\newcommand{\UMLtwo}{\acronym{UML 2.1}}
\newcommand{\JAVA}{\acronym{JAVA}}
\newcommand{\cpp}{\acronym{C++}}
\newcommand{\C}{\acronym{C}}
\newcommand{\HSQL}{\acronym{HSQLDB}}
\newcommand{\SQL}{\acronym{SQL}}
\newcommand{\code}[1]{\foreignlanguage{english}{\texttt{#1}}}
%
% Befehle für den Index
\newcommand{\highlight}[1]{\emph{#1}\index{#1}}
\newcommand{\indexwrite}[1]{#1\index{#1}}
%
% Grafikbefehle für Softwaremetriken
\newcommand{\qualityPoint}[5]{\put(#1,#2){\circle*{2}} \put(#3,#4){\makebox(0,0){\scriptsize{(#5)}}}}

%
% WORKAROUND, damit lstlistoflistings funktioniert:
% Quelle: http://www.komascript.de/node/477
%
%\makeatletter% --> De-TeX-FAQ
%\renewcommand*{\lstlistoflistings}{%
%\begingroup
%\if@twocolumn
%\@restonecoltrue\onecolumn
%\else
%\@restonecolfalse
%\fi
%\lol@heading
%\setlength{\parskip}{\z@}%
%\setlength{\parindent}{\z@}%
%\setlength{\parfillskip}{\z@ \@plus 1fil}%
%\@starttoc{lol}%
%\if@restonecol\twocolumn\fi
%\endgroup
%}
%\makeatother% --> \makeatletter

% Direktes Einbinden von mit Metapost erzeugten Grafiken ermöglichen
\DeclareGraphicsRule{*}{mps}{*}{}

% Codeformatierung
\lstset{language=Java}
\lstset{basicstyle=\small}
\definecolor{darkgrey}{rgb}{0.95,0.95,0.95}
\lstset{backgroundcolor=\color{darkgrey}}
\lstset{linewidth=\textwidth, showstringspaces=false}
\lstset{captionpos=b}
\lstset{tabsize=2}
\lstset{breaklines=true}

% Eigene Commands 
\newcommand{\eigenname}[1]{\textit{#1}}