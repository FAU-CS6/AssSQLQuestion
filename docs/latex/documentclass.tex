% Package:          koma-script
% Documentation:    CTAN:macros/latex/contrib/koma-script
% Papierformat:     A4
% Schriftgröße      12pt
% Documentclass:    scrbook -- alternativer Dokuementtyp zu book mit kleinerem Satzspiegel
%
% Optionen
% ========
% draft / final:    Unterscheidung zwischen Entwurfsversion (draft) und Endversion (final) eines 
%                   Dokuments. Die Option Draft aktiviert schwarze Kästchen an zu langen Zeilen
%                   und gibt anstatt der Abbildungen Platzhalter für die Abbildungen aus.
%                   Für die Endversion des Dokuments muss die Option "final" verwendet werden.
%
% Inhaltsverzeichnis
% ==================
% bibtotoc:         Literaturverzeichnis im Inhaltsverzeichnis mit aufführen
% liststotoc:       Tabellen und Abbildungsverzeichnis im Inhaltsverzeichnis mit aufführen
% idxtotoc:         Index im Inhaltsverzeichnis mit aufführen
% tocleft:          Keine Einrückung im Inhaltsverzeichnis für Abschnitte
% listsleft:        Keine Einrückungen Tabellen- und Abbildungsverezeichnis
% chapterprefix:    Verwende Kapitelnummern
% appendixprefix:   Verwende Kapitelnummern für Anhang
%
% Seitenlayout
% ============
% BCOR:             Korrektur zum Binden des Dokuments in einen Einband
% headsepline:      Trennlinie zwischen Seitenkopf und Textkörper
% halfparskip*:     Verwende anstatt Einrückungen bei einem neuen Absatz 1/4 einer halben
%                   Zeile = 1/8 Zeile vertikalen Abstand.
% halfparskip+:     Verwende anstatt Einrückungen bei einem neuen Absatz 1/3 einer halben
%                   Zeile = 1/6 Zeile vertikalen Abstand.
% 
% Tabellen- und Bildunterschriften
% ================================
% noonelinecaption: Einzeilige Bildunterschriften nicht zentiert, sondern linksbündig
%                   => keine Sonderbehandlung einzeiliger Bild- / Tabellenunterschriften
% origlongtable:    Keine Redefinition der Bildunterschriften des longtable Pakets
%
% Hier wird möglichst alles ausgeschaltet, da ein besseres Paket (caption) für Unterschriften
% verwendet wird.
%
%
% FIXME: Falls Dokument zu lang, option "openany" verwenden. Dann darf ein Kapitel nicht nur auf einer
% Seite rechts open anfangen. Die linke leere Seite fällt dann weg.
% FIXME: Am Ende der Arbeit die Option "draft" auf "final" umstellen
%