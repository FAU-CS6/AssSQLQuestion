%%%%%%%%%%%%
% Tabellen %
%%%%%%%%%%%%
% Paket supertabular: Aufteilung von Tabellen auf mehrere Seiten bzw.
% Einbindung mehrseitige Tabellen. Die sind dann auch wirklich da, wo man sie vom Layout her hinsetzt :-)
\usepackage{multicol}
\usepackage{supertabular}
%
% Paket colortbl: Farbige Tabellenspalten
% Documentation:  CTAN:macros/latex/contrib/colortbl
\usepackage{colortbl}
%
% Paket booktabs: Trennlinien für Tabellen
\usepackage{booktabs}
%
%%%%%%%%%%%%%%%%%%%%%%%%%
% Programmcode Listings %
%%%%%%%%%%%%%%%%%%%%%%%%%
\usepackage{listings}
%
%%%%%%%%%
% Index %
%%%%%%%%%
% Paket makeidx: Zur Erstellung eines Index
\usepackage{makeidx}
% Erstellt einen Index
\makeindex
%
%%%%%%%%%%%%%%%%%%%%%%%%%%%%%%
% Bessere Bildunterschrfiten %
%%%%%%%%%%%%%%%%%%%%%%%%%%%%%%
\usepackage[justification=centerlast,format=hang,font=small,labelfont=bf]{caption}
%
%%%%%%%%%
% Farbe %
%%%%%%%%%
\usepackage{color}
%
%%%%%%%%%%%%
% Grafiken %
%%%%%%%%%%%%
% PDF als Grafiken einbinden
% Intern lädt dieses Paket das Paket graphicx automatisch.
%\usepackage[final]{pdfpages}
% Grafiken einbinden
\usepackage{graphicx}
% LaTeX-Inline-Grafiken
\usepackage{tikz}
\usetikzlibrary{shapes.multipart,shapes.symbols,positioning,calc,arrows}
% Mehrere Grafiken in einer figure Umgebung
%\usepackage{subfigure}
%
%%%%%%%%%%%%%%
% Mathematik %
%%%%%%%%%%%%%%
\usepackage{amsmath}
%
%%%%%%%%%%%%%%
% Hyperlinks %
%%%%%%%%%%%%%%
% Achtung: Das Paket hyperref ist etwas kritisch. Es redefiniert viele Befehle. Es sollte daher nach allen
% anderen Paketen geladen werden und man sollte darauf achten, dass einige Befehle evtl. nicht so wie 
% erwartet funktionieren.
% Dokumentation zu dieser Problematik: /usr/share/doc/texmf/latex/hyperref/README.gz aus dem Paket tetex-doc
%
\usepackage[pdftex,pdfpagelabels,breaklinks=true,pageanchor=true,plainpages=false,pdftitle={\thesistitle},pdfauthor={\myname},pdfsubject={\thesissubject},pdfkeywords={\thesiskeywords},hyperindex=true,colorlinks=false,pdfborder={0 0 0}]{hyperref}

%%%%%%%%%%
% Andere %
%%%%%%%%%%
\usepackage{verbatim}

%%%%%%%%%%%%%%%%%%%%%%%%%
% Biblatex statt Natbib %
%%%%%%%%%%%%%%%%%%%%%%%%%
\usepackage[backend=biber, style=numeric, citestyle=numeric-comp]{biblatex}