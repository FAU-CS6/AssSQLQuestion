\chapter*{Abstract}
\section*{\thesistitleGB}
\selectlanguage{english}

The Open Source Learning Management System ILIAS is a an extensive tool to support students at universities learning. One big feature of ILIAS is the testing module. Using this lecturers are enabled to give their students a test they have to solve before accessing important documents. Of course the testing module can be used to create real (online) exams, too.

While the testing module supports a bunch of different question types by default, including Single Choice, Multiple Choice and Ordering questions, more complex question types have to be added by installing a plugin.

In this project - called \textit{\textbf{Q}uestion \textbf{P}lugin for \textbf{I}lias - \textbf{SQL} \textbf{(QPI-SQL)}} - we developed an ILIAS plugin, named \textit{assSQLQuestion}. The plugin enables lecturers to add interactive SQL questions to their tests. Students can execute their answers and their work is graded automatically by applying different metrics.
 
\clearpage{\pagestyle{empty}\cleardoublepage}
\chapter*{Kurzfassung}
\section*{\thesistitle}
\selectlanguage{ngerman}

Das Open Source Lernmanagement System ILIAS ist ein umfangreiches Werkzeug, um Studenten an Universitäten beim Lernen zu unterstützen. Eine der großen Funktionen von ILIAS ist das Test Modul. Mit Hilfe dieses Moduls können Dozenten Test erstellen, die ihre Studenten beispielsweise lösen müssen, um wichtige Dokumente freizuschalten. Natürlich bietet das Modul auch die Möglichkeit richtige Prüfungen zu erstellen, die dann online stattfinden.

Während das Test Modul von ILIAS nativ bereits eine große Anzahl an verschiedenen Fragetypen unterstützt - wie zum Beispiel Single Choice, Multiple Choice und Sortierungsfragen -, müssen komplexere Fragetypen nachträglich mit Hilfe von Plugins hinzugefügt werden.

In diesem Projekt - genannt \textit{\textbf{Q}uestion \textbf{P}lugin for \textbf{I}lias - \textbf{SQL} \textbf{(QPI-SQL)}} - wurde ein ILIAS Plugin - genannt \textit{assSQLQuestion} - für diesen Zweck entwickelt. Das Plugin ermöglicht es dabei Dozenten interaktive Fragen zum Themenbereich SQL zu stellen. Studenten können dabei ihre Antworten vor der Abgabe ausführen. Die Abgaben werden dabei außerdem automatisch - durch das Anwenden verschiedener Metriken - bewertet.
