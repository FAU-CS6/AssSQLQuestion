Seit 1998 entwickelt der eingetragene Verein \eigenname{ILIAS open source e-Learning e.V.} das Open Source Lernmanagement System \eigenname{ILIAS}. Ziel ist es dabei unter Anderem den digitalen Alltag von Dozenten und Studenten an Universitäten zu vereinfachen, indem Kursmanagement, Lernmodule, Tests und Prüfungen unter einer einzigen Plattform zusammengeführt werden (s. \cite{Ilias1998}). Während das Kursmanagement und die Lernmodule vor allem dazu dienen Studenten Informationen, Skripte und Übungsblätter bereitzustellen, bringen Tests und Prüfungen einen interessanten Aspekt der Interaktion in das Lernmanagement System. Dozenten können Tests dazu nutzen Studenten eine freiwillige Möglichkeit des Selbsttests zu bieten. Sie können aber auch die Verfügbarkeit von Materialien für die Studenten daran koppeln, dass bestimmte Tests zunächst von diesen bearbeitet und bestanden werden müssen. Mit Hilfe der Tests können ganze Prüfungen, sogenannte E-Klausuren, in \eigenname{ILIAS} abgehalten werden, was den Korrekturaufwand erheblich mindert.

Das \eigenname{Insitut für Lerninnovation} \eigenname{(ILI)}, der \eigenname{Friedrich-Alexander Universität Erlangen-Nürnberg} \eigenname{(FAU)}, hat im Sommersemester 2007 damit begonnen ILIAS für den Betrieb an der \eigenname{FAU} vorzubereiten und unter dem Eigennamen \eigenname{StudOn} anzupassen (s. \cite{Studon2007}). Damit wurden auch an der \eigenname{FAU} die umfangreichen Möglichkeiten von \eigenname{ILIAS} für Dozenten und Studenten eingeführt. Dies beinhaltet auch die Möglichkeiten, die \eigenname{ILIAS} bei Tests und Prüfungen bietet. 

Ein großer Nachteil dieser Tests ist es, dass \eigenname{ILIAS} von Haus aus nur eine begrenzte Anzahl an Fragentypen für die Tests anbietet. So sind beispielsweise Einfachantwort-, Mehrfachantwort- und Freitextfragen möglich (s. \cite{IliasAutorenDokumentation}, Kapitel 14.3), komplexere Fragetypen werden aber nur begrenzt unterstützt. Dem \eigenname{Lehrstuhl für Informatik 6 (Datenmanagement)} stellte sich hier die Frage, wie sich am besten Fragen zur Datenbankanfragesprache \eigenname{SQL} stellen lassen. Eine der bereits möglichen Umsetzungen wäre es eine Freitextanfrage zu nutzen. Diese Möglichkeit bringt aber zwei entscheidende Nachteile mit sich: Weder können Studenten das SQL ausführen, um eine Prüfungssituation nah an der Praxis zu erhalten, noch können die Antworten automatisch bewertet und korrigiert werden.

Da \eigenname{ILIAS} und damit auch \eigenname{StudOn} die Möglichkeit bieten, Plugins für neue Fragetypen zu entwickeln, entstand die Idee ein solches Plugin speziell für Fragen zur Anfragesprache \eigenname{SQL} zu entwickeln. Ein Plugin das es Studenten ermöglicht \eigenname{SQL} während des Tests auszuführen und den Dozenten über Metriken eine automatische Bewertung der Abgaben bietet. Dieses Plugin entstand als Teil des \eigenname{\textbf{Q}uestion \textbf{P}lugin for \textbf{I}lias - \textbf{SQL} \textbf{(QPI-SQL)}} Projektes.