Seit 1998 entwickelt der eingetragene Verein \textit{ILIAS open source e-Learning} das Open Source Lernmanagement System \textit{ILIAS}. Ziel ist es dabei unter anderem, den digitalen Alltag von Dozenten und Studenten an Universitäten zu vereinfachen, in dem Kursmanagement, Lernmodule, Tests und Prüfungen unter einer einzigen Plattform zusammengeführt werden (s. \cite{Ilias1998}). Während das Kursmanagement und die Lernmodule vor allem dazu dienen, Studenten Informationen, Scripte und Übungsblätter bereitzustellen, bringen Tests und Prüfungen einen interessanten Aspekt der Interaktion in das Lernmanagement System. Dozenten können Tests dazu nutzen Studenten eine freiwillige Möglichkeit des Selbsttests zu bieten oder aber die Verfügbarkeit von Materialien für die Studenten daran koppeln, dass bestimmte Tests zunächst von diesen bearbeitet und bestanden werden müssen. Auch können mit Hilfe der Tests ganze Prüfungen, sogenannte E-Klausuren, in \textit{ILIAS} abgehalten werden, was den Korrekturaufwand erheblich mindern kann.

Das \textit{Insitut für Lerninnovation} \textit{(ILI)}, der \textit{Friedrich-Alexander Universität Erlangen-Nürnberg} \textit{(FAU)}, hat im Sommersemester 2007 damit begonnen ILIAS für den Betrieb an der \textit{FAU} vorzubereiten und unter dem Eigennamen \textit{StudOn} anzupassen (s. \cite{Studon2007}). Damit wurden auch an der \textit{FAU} die umfangreichen Möglichkeiten von \textit{ILIAS} für Dozenten und Studenten eingeführt. Dies beinhaltet auch die Möglichkeiten, die \textit{ILIAS} bei Tests und Prüfungen bietet. 

