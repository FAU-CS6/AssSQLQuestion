\chapter*{Kurzfassung}
\section*{\thesistitle}
\selectlanguage{ngerman}

Das Open Source Lernmanagement System ILIAS ist ein umfangreiches Werkzeug, um Studenten an Universitäten beim Lernen zu unterstützen. Eine der großen Funktionen von ILIAS ist das Test Modul. Mit Hilfe dieses Moduls können Dozenten Tests erstellen, die ihre Studenten beispielsweise lösen müssen, um wichtige Dokumente freizuschalten. Natürlich bietet das Modul auch die Möglichkeit Prüfungen zu erstellen, die dann online stattfinden.

Während das Test Modul von ILIAS nativ bereits eine große Anzahl an verschiedenen Fragetypen unterstützt - wie zum Beispiel Single Choice, Multiple Choice und Sortierungsfragen -, müssen komplexere Fragetypen nachträglich mit Hilfe von Plugins hinzugefügt werden.

In diesem Projekt - genannt \eigenname{\textbf{Q}uestion \textbf{P}lugin for \textbf{I}lias - \textbf{SQL} \textbf{(QPI-SQL)}} - wurde ein ILIAS Plugin - genannt \eigenname{assSQLQuestion} - für diesen Zweck entwickelt. Das Plugin ermöglicht Dozenten interaktive Fragen zum Themenbereich SQL zu stellen. Studenten können dabei ihre Antworten vor der Abgabe ausführen. Die Abgaben werden dabei außerdem automatisch - durch das Anwenden verschiedener Metriken - bewertet. 